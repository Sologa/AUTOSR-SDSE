\section{Survey of representative \slms}

\label{sec:lit_review}
The previous sections have described the components, design choices, and training strategies for \slms.  In this section, we %
review how these choices have been instantiated in existing \slms\ in the literature, 
categorized according to the classes in Table~\ref{tab:typology}.
For each category, we provide a brief review of the model design and highlight example models in the literature.
Note that, aside from the examples listed, there are many \slms\  with similar configurations in each category which we are unable to introduce in detail; \Cref{table:SLMs} provides a more extensive catalogue of models and references, and~\Cref{fig:timeline} provides a timeline to place them in historical context.

We also note that some of the 
multi-modal LLMs developed by industry groups, such as GPT-4o~\citep{gpt4osafety} and Gemini~\citep{geminiteam2024gemini15unlockingmultimodal}, can also be seen as generalizations of speech+text LMs or speech-aware text LMs.
These models have 
conversation capabilities and include text, speech, general audio, and visual inputs and outputs.
However, details of their model configurations and training strategies have not been publicly released, and rigorous evaluations of these models have focused on limited speech tasks such as ASR and ST.
For these reasons, it is important that future work complements these high-impact industry models with open-source, reproducible approaches.

\subsection{Pure Speech LM}
\label{sec:pureSLM}
Pure speech LMs
are trained on unlabelled speech data to model $p(speech)$, and are the spoken analogue of pre-trained generative text LMs like GPT~\citep{radford2019language, brown2020language}.
In order to apply the next-token-prediction objectives of LM training to continuous speech data, pure speech LMs typically use discrete speech representations (\Cref{sec:speech-encoder}) as the units for sequence modeling.  
One of the earliest approaches in this category is the generative spoken language modeling (GSLM) work of~\citet{GSLM}. %
GSLM relies on a pre-trained HuBERT model $\text{Enc}^{\text{sp}}()$ to convert speech signals into phonetic tokens $H^{\text{sp}}$ and uses an embedding layer $\text{Adp}^{\text{sp}}$() to map token IDs to continuous vectors.
The model is trained to autoregressively predict the ID of the next speech token, and the predicted tokens are converted back to audio by a unit-based vocoder $\text{Dec}^{\text{sp}}()$.
GSLM's demo\footnote{https://speechbot.github.io/gslm} shows anecdotally that this initial model can generate speech continuations with locally acceptable syntax given a speech prompt of a few seconds, but does not in general produce coherent long-form speech.

Since GSLM initiated this line of research, follow-up work has improved on it in several ways. %
For example, the prosody-aware generative spoken language model (pGSLM)~\citep{pGSLM} 
predicts prosodic information like pitch and duration jointly with phonetic tokens for better expressiveness.
TWIST~\citep{twist} is similar to GSLM but the model is initialized with a pre-trained text LLM (OPT~\citep{zhang2022opt}), which improves lexical and syntactic language modeling performance, as measured by the sWUGGY and sBLIMP metrics (see~\Cref{ssec:likelihood_eval} for more details of the evaluation).
Another approach~\citep{tGSLM} uses \emph{longer} lexical units instead of phonetic tokens,
significantly reducing memory consumption. %
AudioLM~\citep{audiolm} introduced hierarchical generation of multiple types of tokens for pure speech LMs, using a coarse-to-fine strategy that generates phonetic tokens and audio codec tokens in order (\Cref{ssec:hybrid_generation}).
SpeechSSM~\citep{park2024long} uses a state-space model~\citep{gu2021efficiently} rather than a Transformer, producing more natural and semantically consistent long-form (minutes-long) speech generation. 


\subsection{Speech+Text LM}
\label{subsubsec:cluster4}
Speech+text LMs~\citep{nachmani2023spoken, nguyen2024spirit, defossezmoshi} are \slms\space that %
jointly model the distribution of speech and text $p(text, speech)$, and can therefore understand and generate both modalities.  Such models often start with generative speech+text pre-training (\Cref{sec:pre-training}), and can then be post-trained in a variety of ways.

A notable example of this approach is the Moshi model \citep{defossezmoshi}, which is the first open-source speech-in-speech-out \slm-based dialogue system with real-time inference capability. 
Moshi takes time-aligned text and discrete speech representations as inputs %
and generates outputs in both text and speech forms.
Moshi’s text processing components---$\text{Enc}^{\text{txt}}()$, $\text{Adp}^{\text{txt}}()$, and $\text{Dec}^{\text{txt}}()$, and the LM backbone $\text{Seq}()$---are initialized from a pre-trained Helium text LLM~\citep{defossezmoshi}.
After text pre-training, the LM is continually trained on jointly predicting the next speech and text tokens, followed by post-training in duplex mode (described in detail in \Cref{sec:duplex}) on conversation data.
Moshi's training strategy produces a model that can generate consistent speech across long (several-minute) monologues and can engage in multi-turn conversations~\citep{defossezmoshi}.

Speech+text LMs vary widely in their %
decoding methods and approaches to constructing speech and text inputs/outputs.
For example, SpiRit-LM~\citep{nguyen2024spirit} uses interleaved sequences of text and discrete speech tokens as both inputs and outputs, focusing solely on pre-training without doing any task-specific post-training.  Compared to pure speech LMs, SpiRiT-LM has improved semantic understanding, as measured by the StoryCloze test (see~\Cref{ssec:likelihood_eval} for more details).
A precursor of this idea, SUTLM~\citep{chou2023toward}, modeled interleaved speech and text units of various sizes, but did not include a waveform synthesizer and was evaluated mainly on spoken language understanding tasks, with the main goal of enabling cross-modality transfer.
\citet{zeng2024scaling} scale up the idea of interleaved speech-text training by using a speech synthesizer to generate massive datasets of interleaved speech-text sequences.
As alternatives to time-aligned or interleaved speech and text sequences, SpeechGPT~\citep{zhang2023speechgpt} outputs a sequence of text %
followed by corresponding speech, while Mini-Omni~\citep{xie2024mini,xie2024mini2} and LLaMA-Omni~\citep{fang2024llama} generate text and speech using separate output channels with a delay pattern in the speech token prediction (see~\Cref{subsubsec:text-speech-hybrid}).
All three of these models (SpeechGPT, Mini-Omni, and LLaMA-Omni), unlike Moshi, focus on a traditional turn-taking structure and therefore have limited capability to model spontaneous turn-taking behavior (see \Cref{sec:duplex} for more on conversation modeling).



\subsection{Speech-aware text LM}
\label{subsec:IF_cluster1}
Models in this category combine text LMs with speech encoders, and usually
take both speech and text as inputs and generate responses in text form.
The text input $X^{\text{txt}}$ can include 
instructions asking the model to 
perform 
tasks related to the speech input $X^{\text{sp}}$, possibly including tasks for which the model is not specifically trained.
This type of \slm\ 
is typically initialized with a 
text LLM (which has often already been post-trained with instruction tuning or RLHF) so as to inherit its linguistic knowledge and instruction-following capabilities.
After combining the text LLM with a speech encoder and modality adapter, it is common to apply training methods that encourage better alignment between the speech and text modalities (\Cref{ssec:alignment}) or to start post-training with an ASR task (\Cref{ssec:training_tricks}) to help extract content information from speech inputs.
The following post-training typically uses speech instruction-tuning datasets with diverse instructions 
(\Cref{ssec:instruction_tuning}). 


WavPrompt~\citep{gao2022wavprompt} represents the first example of this line of research.  WavPrompt combines a wav2vec 2.0~\citep{wav2vec2} SSL speech encoder with a GPT-2 text LM and is trained for speech classification tasks, keeping the LM backbone frozen and updating only the speech encoder.
Although WavPrompt can improve over ASR+text model baselines on certain tasks, it is not evaluated on any unseen task.
Since then, many models in this category have been proposed. 

SALMONN~\citep{tang2023salmonn} is another early and impactful approach.
The model is built on a pre-trained Vicuna LM backbone~\citep{zheng2023judging}.
During post-training,
the model is first trained on ASR and audio captioning and later on a more diverse set of speech understanding tasks.
SALMONN's post-training keeps most of the parameters frozen---only the speech modality adapter and LoRA~\citep{lora} adapters for the LM are learned---%
but the model is still able to generalize to unseen tasks with
various natural language prompts.

Several other speech-aware text LMs were developed nearly simultaneously with SALMONN, facilitated by the release of open-source LLMs like LLaMA~\citep{touvron2023llama} and Vicuna~\citep{zheng2023judging}, often post-trained and evaluated on different tasks~\citep{gong2023joint,wang2023slm,chu2023qwen,chen2024salm}.  
For example, LTU-AS~\citep{gong2023joint} combines  spoken content understanding with paralinguistic analysis tasks (e.g., emotion recognition).
SLM~\citep{wang2023slm} generalizes to unseen tasks specified by natural language prompts even while the LM backbone and speech encoder are both frozen, and only the speech adapter is optimized during instruction-based training, but focuses only on content-heavy tasks.
Qwen-Audio~\citep{chu2023qwen} shares similar post-training tasks to those of SALMONN, but includes learning of the speech encoder and LM backbone (sequence model) weights.
During post-training, Qwen-Audio starts with a more diverse task set instead of the single ASR task,
including both content and paralinguistics, followed by a chat-based supervied fine-tuning stage that improves the model's robustness to variation in input prompts.
The follow-up work, Qwen2-Audio~\citep{chu2024qwen2audio} also applies direct preference optimization~\cite{rafailov2023direct} after supervised fine-tuning.
On the other hand, the more recent DeSTA~\citep{desta} and DeSTA 2~\citep{desta2} propose a single descriptive speech-text alignment post-training task which requires the model to both recognize the spoken content and describe paralinguistic attributes.  \Cref{tab:universal-models-2} summarizes and compares a representative set of these models' training and evaluation choices.  One key consideration is the tradeoff between preserving knowledge acquired through text pre-training and learning new speech tasks, especially those requiring non-textual information such as speech emotion.  An important design choice, therefore, is which of the pre-trained parameters to freeze or update, and this issue has not yet been thoroughly explored.

In addition to the multiple post-training approaches, speech-aware text LMs also include some differences in architecture choices.  For example, WavLLM~\citep{wavllm} combines both SSL and supervised pre-trained speech encoders, and BESTOW~\citep{bestow2024chen} experiments with a sequence-to-sequence model in addition to the common decoder-only LM architecture.
While most models in this category are initialized with pre-trained text LLMs, UniverSLU ~\citep{arora2023universlu} 
starts from a Whisper model for conditional pre-training (see \Cref{ssec:conditional_pretraining})
and is then fine-tuned for various understanding tasks first by prepending new task specifiers and then by adding natural language instructions to the input.






